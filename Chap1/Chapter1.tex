\chapter{Introduction}
\label{chap:Intro}

\section{Background and Motivation}
Sentiment analysis, also known as opinion mining, is a pivotal sub-domain of Natural Language Processing (NLP) that deals with the computational identification, extraction, and categorization of opinions, sentiments, and emotions expressed in text. It has evolved from a simple binary classification task to a complex field capable of understanding sarcasm, irony, and context-dependent sentiments. In the era of digital information, the volume of news and social media content has exploded, making manual monitoring impossible. Every second, millions of data points are generated across various platforms, necessitating automated systems to process and interpret this vast amount of information.

For a developing nation like Bangladesh, analyzing the sentiment of news media is not merely a technical challenge but a necessity for understanding the socio-political landscape. News media serves as a mirror to society, reflecting public reaction to government policies, monitoring social unrest, and highlighting economic trends. An automated system capable of gauging the sentiment of news articles can provide invaluable insights to policymakers, journalists, and social scientists. For instance, tracking the sentiment trends regarding a new educational policy can help the government gauge public acceptance or resistance early on. Similarly, monitoring reports on price hikes or commodity shortages can serve as an early warning system for potential economic instability.

While English NLP has reached near-human performance with the advent of Large Language Models (LLMs) like GPT-4 and Llama, low-resource languages like Bengali (Bangla) lag significantly behind. Bengali is the seventh most spoken language globally, with over 230 million native speakers spread across Bangladesh and India. Despite this massive demographic, it remains under-represented in the digital NLP landscape. The challenges are manifold:
\begin{enumerate}
    \item \textbf{Morphological Richness:} Bengali is a highly inflected language. A single root word can have dozens of variations based on tense, person, case, and gender. For example, the root word "কর" (do) can manifest as "করছি" (doing), "করেছিলাম" (did), "করব" (will do), "করাচ্ছে" (making someone do), and so on. This morphological complexity leads to a sparse data problem, where many valid word forms may not appear in the training data.
    \item \textbf{Complex Orthography:} The Bengali script uses abugidas, where vowels are often written as diacritics attached to consonants. It also features numerous conjunct characters (juktakkhor), which pose challenges for tokenization and character-level encoding.
    \item \textbf{Resource Scarcity:} There is a severe lack of large-scale, annotated datasets for specific domains like formal news. Most available datasets are small ($\approx$ 5-10k samples) or focus on informal social media comments, which differ significantly in grammar and vocabulary from formal news reportage \cite{sarker2021benchmarking}.
    \item \textbf{Diglossia:} Bengali exhibits a strong diglossia, meaning there is a significant difference between the formal written language (Sadhu/Cholitobhasha) used in news and the informal spoken dialects. Models trained on casual social media text often fail to generalize to the formal register of news articles.
\end{enumerate}

This thesis addresses these gaps by developing a robust, end-to-end sentiment analysis system specifically tailored for Bengali news articles. By leveraging state-of-the-art multilingual transformer models and fine-tuning them on a newly curated, large-scale dataset, this research aims to bridge the performance gap between English and Bengali NLP.

\section{Problem Statement}
The core problem addressed in this research is the \textit{lack of a scalable, automated, and domain-adapted pipeline for analyzing the sentiment of formal Bengali news}. 

Existing solutions primarily suffer from the following limitations:
\begin{itemize}
    \item \textbf{Reliance on Lexicon-based Approaches:} Traditional methods rely on sentiment dictionaries. These fail to capture context. For instance, the word "জটিল" (complex) can be negative in "সমস্যাটি জটিল" (The problem is complex) but positive in "জটিল শট" (An amazing shot) in a sports context. Lexicon-based models cannot distinguish these nuances.
    \item \textbf{Loss in Translation:} A common practice is to translate Bengali text to English using Google Translate API and then apply English sentiment models. This "Translation-based method" introduces significant latency and cost. More critically, it leads to a loss of cultural nuance and idiomatic meaning. A phrase like "ঘোড়ার ডিম" (literally: horse's egg) means "nonsense" or "nothing", but a translation system might interpret it literally, missing the negative sentiment entirely.
    \item \textbf{Inadequacy of Traditional ML:} Shallow Machine Learning models like Support Vector Machines (SVM) and Naive Bayes require extensive manual feature engineering (e.g., n-grams, TF-IDF). These features often fail to capture semantic relationships and long-range dependencies in text.
    \item \textbf{Domain Mismatch in Pre-trained Models:} While multilingual models like mBERT and XLM-RoBERTa exist, they are pre-trained on diverse corpora (like Wikipedia) that may not align with the specific vocabulary and style of Bengali journalism. Without fine-tuning, their performance on specific downstream tasks remains sub-optimal.
\end{itemize}

Furthermore, recent works in 2024 have begun to explore transformers \cite{rahman2024enhancing}, but they often lack a comprehensive, open-access, large-scale dataset of full-length news articles, relying instead on headlines or short social media posts. This thesis seeks to solve the problem of analyzing \textit{full-length documents} by effectively utilizing truncation and focused fine-tuning strategies.

\section{Research Objectives}
The primary goal of this thesis is to build a high-performance, deep-learning-based sentiment analysis system for Bengali news. To achieve this, the following specific objectives have been defined:

\begin{enumerate}
    \item \textbf{Large-Scale Data Collection:} To design and implement a distributed web scraping pipeline capable of aggregating thousands of articles from diverse sources, ensuring representation from various domains (politics, sports, entertainment, crime).
    \item \textbf{Dataset Curation and Cleaning:} To process the raw scraped data, addressing issues like Unicode normalization, duplicate removal, and noise reduction. The objective is to create a clean, semi-supervised labeled dataset of over 70,000 news samples, significantly larger than many existing benchmarks.
    \item \textbf{Transfer Learning Application:} To investigate the efficacy of Transfer Learning by fine-tuning a \texttt{Pre-trained Multilingual DistilBERT} model. The objective is to adapt the model's pre-existing multilingual knowledge to the specific context of Bengali news.
    \item \textbf{Performance Evaluation:} To rigorously evaluate the model using standard metrics (Accuracy, Precision, Recall, F1-Score) and compare it against baseline models (mBERT, XLM-RoBERTa) to demonstrate the improvement achieved through fine-tuning.
    \item \textbf{System Development and Deployment:} To deploy the fine-tuned model as a user-friendly web application using Gradio, allowing for real-time inference and batch processing, thereby making the research accessible to end-users without technical expertise.
\end{enumerate}

\section{Scope and Limitations}
\subsection{Scope}
This research focuses on:
\begin{itemize}
    \item \textbf{Text-based Analysis:} The system analyzes only the textual content of news articles. Images and videos are outside the scope of this study.
    \item \textbf{Formal Bengali:} The dataset and model are tailored for formal definitions of standard Bengali (Cholitobhasha) as used in major newspapers.
    \item \textbf{Document-level Sentiment:} The system classifies the overall sentiment of an article (Positive, Negative, or Neutral). It does not perform aspect-based sentiment analysis (e.g., identifying sentiment towards specific entities within the text).
\end{itemize}

\subsection{Limitations}
Despite the rigorous methodology, this study has certain limitations:
\begin{itemize}
    \item \textbf{Dependency on Weak Labels:} Due to the prohibitive cost of manual annotation for 70,000 articles, we utilized a pre-trained model for initial labeling (weak supervision). While effective, this creates a dependency on the teacher model's accuracy and may introduce some noise into the training data.
    \item \textbf{Input Length Constraint:} The DistilBERT architecture imposes a maximum sequence length of 512 tokens. News articles exceeding this length are truncated, which may result in the loss of crucial information present at the end of long reports.
    \item \textbf{Sarcasm Detection:} Detecting high-level sarcasm or political satire remains a challenge for the model, as these rely on deep cultural knowledge and external context often missing from the text itself.
\end{itemize}

\section{Thesis Contributions}
The key contributions of this research are:
\begin{itemize}
    \item \textbf{Curated Dataset:} Creation of a large-scale, cleaner, and semi-supervised labeled dataset of 70,300+ Bengali news articles, which serves as a valuable resource for future research in this domain.
    \item \textbf{State-of-the-art Accuracy:} Achieving a 93\% accuracy through fine-tuning, significantly outperforming standard mBERT and lexicon-based baselines for this specific task.
    \item \textbf{Open Source Tool:} Development and release of an open-source web application with visualization capabilities, bridging the gap between theoretical research and practical application.
\end{itemize}

\section{Thesis Organization}
The remainder of this thesis is organized as follows:
\begin{itemize}
    \item \textbf{Chapter 2 (Literature Review)} provides a comprehensive review of related work, tracing the history of sentiment analysis from lexicon-based methods to modern Transformer architectures, with a specific focus on Bengali literature.
    \item \textbf{Chapter 3 (Theoretical Background and Methodology)} details the theoretical foundations of the Transformer model, including the self-attention mechanism, and describes the proposed methodology for data collection and model training.
    \item \textbf{Chapter 4 (System Implementation)} presents the implementation details, describing the software architecture, the scraping pipeline, and the development of the web interface.
    \item \textbf{Chapter 5 (Results and Analysis)} discusses the experimental results, provides a comparative analysis with baselines, and offers a visual and error analysis of the model's performance.
    \item \textbf{Chapter 6 (Conclusion)} concludes the thesis by summarizing the findings and outlining potential directions for future research.
\end{itemize}
