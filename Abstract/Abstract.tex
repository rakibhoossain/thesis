

Sentiment analysis, the computational study of people's opinions, sentiments, evaluations, attitudes, and emotions from written language, has become a pivotal field in Natural Language Processing (NLP). While significant progress has been made for high-resource languages like English, low-resource languages such as Bengali continue to face challenges due to morphological complexity, diglossia, and a scarcity of large-scale, annotated datasets. This thesis addresses these gaps by presenting a comprehensive framework for \textit{Automated Sentiment Analysis of Bangla News} using state-of-the-art Deep Learning techniques.

We constructed a robust, domain-specific dataset by scraping and curating over 70,300 news articles from leading Bengali publications, including \textit{Prothom Alo}, \textit{Kalerkantho}, and \textit{BBC Bangla}. To overcome the bottleneck of manual annotation, we employed a semi-supervised learning strategy, utilizing a pre-trained multilingual "Teacher" model to generate weak labels, which were then mapped to a robust 3-class sentiment schema (Positive, Negative, Neutral).

The core of our contribution lies in the fine-tuning of a \textbf{Multilingual DistilBERT} transformer model. By leveraging the transfer learning capabilities of this architecture, we adapted its pre-trained knowledge to the specific linguistic nuances of the Bengali news domain. Our proposed model achieved a test accuracy of \textbf{93.1\%}, demonstrating superior performance over standard baselines such as mBERT (68.2\%) and XLM-RoBERTa (72.5\%).

Furthermore, to bridge the gap between research and practice, we developed a fully functional web-based inference engine using Gradio. This application allows for both real-time single-text analysis and bulk CSV processing, providing visualization of sentiment distribution and confidence calibration. This thesis not only contributes a high-performance model but also establishes a reproducible pipeline for low-resource language sentiment analysis, offering a valuable resource for media monitoring, social science research, and public opinion tracking in Bangladesh.
